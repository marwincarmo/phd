% Options for packages loaded elsewhere
\PassOptionsToPackage{unicode}{hyperref}
\PassOptionsToPackage{hyphens}{url}
\PassOptionsToPackage{dvipsnames,svgnames,x11names}{xcolor}
%
\documentclass[
  letterpaper,
  DIV=11,
  numbers=noendperiod]{scrartcl}

\usepackage{amsmath,amssymb}
\usepackage{iftex}
\ifPDFTeX
  \usepackage[T1]{fontenc}
  \usepackage[utf8]{inputenc}
  \usepackage{textcomp} % provide euro and other symbols
\else % if luatex or xetex
  \usepackage{unicode-math}
  \defaultfontfeatures{Scale=MatchLowercase}
  \defaultfontfeatures[\rmfamily]{Ligatures=TeX,Scale=1}
\fi
\usepackage{lmodern}
\ifPDFTeX\else  
    % xetex/luatex font selection
\fi
% Use upquote if available, for straight quotes in verbatim environments
\IfFileExists{upquote.sty}{\usepackage{upquote}}{}
\IfFileExists{microtype.sty}{% use microtype if available
  \usepackage[]{microtype}
  \UseMicrotypeSet[protrusion]{basicmath} % disable protrusion for tt fonts
}{}
\makeatletter
\@ifundefined{KOMAClassName}{% if non-KOMA class
  \IfFileExists{parskip.sty}{%
    \usepackage{parskip}
  }{% else
    \setlength{\parindent}{0pt}
    \setlength{\parskip}{6pt plus 2pt minus 1pt}}
}{% if KOMA class
  \KOMAoptions{parskip=half}}
\makeatother
\usepackage{xcolor}
\setlength{\emergencystretch}{3em} % prevent overfull lines
\setcounter{secnumdepth}{-\maxdimen} % remove section numbering
% Make \paragraph and \subparagraph free-standing
\ifx\paragraph\undefined\else
  \let\oldparagraph\paragraph
  \renewcommand{\paragraph}[1]{\oldparagraph{#1}\mbox{}}
\fi
\ifx\subparagraph\undefined\else
  \let\oldsubparagraph\subparagraph
  \renewcommand{\subparagraph}[1]{\oldsubparagraph{#1}\mbox{}}
\fi

\usepackage{color}
\usepackage{fancyvrb}
\newcommand{\VerbBar}{|}
\newcommand{\VERB}{\Verb[commandchars=\\\{\}]}
\DefineVerbatimEnvironment{Highlighting}{Verbatim}{commandchars=\\\{\}}
% Add ',fontsize=\small' for more characters per line
\usepackage{framed}
\definecolor{shadecolor}{RGB}{241,243,245}
\newenvironment{Shaded}{\begin{snugshade}}{\end{snugshade}}
\newcommand{\AlertTok}[1]{\textcolor[rgb]{0.68,0.00,0.00}{#1}}
\newcommand{\AnnotationTok}[1]{\textcolor[rgb]{0.37,0.37,0.37}{#1}}
\newcommand{\AttributeTok}[1]{\textcolor[rgb]{0.40,0.45,0.13}{#1}}
\newcommand{\BaseNTok}[1]{\textcolor[rgb]{0.68,0.00,0.00}{#1}}
\newcommand{\BuiltInTok}[1]{\textcolor[rgb]{0.00,0.23,0.31}{#1}}
\newcommand{\CharTok}[1]{\textcolor[rgb]{0.13,0.47,0.30}{#1}}
\newcommand{\CommentTok}[1]{\textcolor[rgb]{0.37,0.37,0.37}{#1}}
\newcommand{\CommentVarTok}[1]{\textcolor[rgb]{0.37,0.37,0.37}{\textit{#1}}}
\newcommand{\ConstantTok}[1]{\textcolor[rgb]{0.56,0.35,0.01}{#1}}
\newcommand{\ControlFlowTok}[1]{\textcolor[rgb]{0.00,0.23,0.31}{#1}}
\newcommand{\DataTypeTok}[1]{\textcolor[rgb]{0.68,0.00,0.00}{#1}}
\newcommand{\DecValTok}[1]{\textcolor[rgb]{0.68,0.00,0.00}{#1}}
\newcommand{\DocumentationTok}[1]{\textcolor[rgb]{0.37,0.37,0.37}{\textit{#1}}}
\newcommand{\ErrorTok}[1]{\textcolor[rgb]{0.68,0.00,0.00}{#1}}
\newcommand{\ExtensionTok}[1]{\textcolor[rgb]{0.00,0.23,0.31}{#1}}
\newcommand{\FloatTok}[1]{\textcolor[rgb]{0.68,0.00,0.00}{#1}}
\newcommand{\FunctionTok}[1]{\textcolor[rgb]{0.28,0.35,0.67}{#1}}
\newcommand{\ImportTok}[1]{\textcolor[rgb]{0.00,0.46,0.62}{#1}}
\newcommand{\InformationTok}[1]{\textcolor[rgb]{0.37,0.37,0.37}{#1}}
\newcommand{\KeywordTok}[1]{\textcolor[rgb]{0.00,0.23,0.31}{#1}}
\newcommand{\NormalTok}[1]{\textcolor[rgb]{0.00,0.23,0.31}{#1}}
\newcommand{\OperatorTok}[1]{\textcolor[rgb]{0.37,0.37,0.37}{#1}}
\newcommand{\OtherTok}[1]{\textcolor[rgb]{0.00,0.23,0.31}{#1}}
\newcommand{\PreprocessorTok}[1]{\textcolor[rgb]{0.68,0.00,0.00}{#1}}
\newcommand{\RegionMarkerTok}[1]{\textcolor[rgb]{0.00,0.23,0.31}{#1}}
\newcommand{\SpecialCharTok}[1]{\textcolor[rgb]{0.37,0.37,0.37}{#1}}
\newcommand{\SpecialStringTok}[1]{\textcolor[rgb]{0.13,0.47,0.30}{#1}}
\newcommand{\StringTok}[1]{\textcolor[rgb]{0.13,0.47,0.30}{#1}}
\newcommand{\VariableTok}[1]{\textcolor[rgb]{0.07,0.07,0.07}{#1}}
\newcommand{\VerbatimStringTok}[1]{\textcolor[rgb]{0.13,0.47,0.30}{#1}}
\newcommand{\WarningTok}[1]{\textcolor[rgb]{0.37,0.37,0.37}{\textit{#1}}}

\providecommand{\tightlist}{%
  \setlength{\itemsep}{0pt}\setlength{\parskip}{0pt}}\usepackage{longtable,booktabs,array}
\usepackage{calc} % for calculating minipage widths
% Correct order of tables after \paragraph or \subparagraph
\usepackage{etoolbox}
\makeatletter
\patchcmd\longtable{\par}{\if@noskipsec\mbox{}\fi\par}{}{}
\makeatother
% Allow footnotes in longtable head/foot
\IfFileExists{footnotehyper.sty}{\usepackage{footnotehyper}}{\usepackage{footnote}}
\makesavenoteenv{longtable}
\usepackage{graphicx}
\makeatletter
\def\maxwidth{\ifdim\Gin@nat@width>\linewidth\linewidth\else\Gin@nat@width\fi}
\def\maxheight{\ifdim\Gin@nat@height>\textheight\textheight\else\Gin@nat@height\fi}
\makeatother
% Scale images if necessary, so that they will not overflow the page
% margins by default, and it is still possible to overwrite the defaults
% using explicit options in \includegraphics[width, height, ...]{}
\setkeys{Gin}{width=\maxwidth,height=\maxheight,keepaspectratio}
% Set default figure placement to htbp
\makeatletter
\def\fps@figure{htbp}
\makeatother

\KOMAoption{captions}{tableheading}
\makeatletter
\makeatother
\makeatletter
\makeatother
\makeatletter
\@ifpackageloaded{caption}{}{\usepackage{caption}}
\AtBeginDocument{%
\ifdefined\contentsname
  \renewcommand*\contentsname{Table of contents}
\else
  \newcommand\contentsname{Table of contents}
\fi
\ifdefined\listfigurename
  \renewcommand*\listfigurename{List of Figures}
\else
  \newcommand\listfigurename{List of Figures}
\fi
\ifdefined\listtablename
  \renewcommand*\listtablename{List of Tables}
\else
  \newcommand\listtablename{List of Tables}
\fi
\ifdefined\figurename
  \renewcommand*\figurename{Figure}
\else
  \newcommand\figurename{Figure}
\fi
\ifdefined\tablename
  \renewcommand*\tablename{Table}
\else
  \newcommand\tablename{Table}
\fi
}
\@ifpackageloaded{float}{}{\usepackage{float}}
\floatstyle{ruled}
\@ifundefined{c@chapter}{\newfloat{codelisting}{h}{lop}}{\newfloat{codelisting}{h}{lop}[chapter]}
\floatname{codelisting}{Listing}
\newcommand*\listoflistings{\listof{codelisting}{List of Listings}}
\makeatother
\makeatletter
\@ifpackageloaded{caption}{}{\usepackage{caption}}
\@ifpackageloaded{subcaption}{}{\usepackage{subcaption}}
\makeatother
\makeatletter
\@ifpackageloaded{tcolorbox}{}{\usepackage[skins,breakable]{tcolorbox}}
\makeatother
\makeatletter
\@ifundefined{shadecolor}{\definecolor{shadecolor}{rgb}{.97, .97, .97}}
\makeatother
\makeatletter
\makeatother
\makeatletter
\makeatother
\ifLuaTeX
  \usepackage{selnolig}  % disable illegal ligatures
\fi
\IfFileExists{bookmark.sty}{\usepackage{bookmark}}{\usepackage{hyperref}}
\IfFileExists{xurl.sty}{\usepackage{xurl}}{} % add URL line breaks if available
\urlstyle{same} % disable monospaced font for URLs
\hypersetup{
  pdftitle={PSC205A Assignment 01: Matrix Algebra},
  colorlinks=true,
  linkcolor={blue},
  filecolor={Maroon},
  citecolor={Blue},
  urlcolor={Blue},
  pdfcreator={LaTeX via pandoc}}

\title{PSC205A Assignment 01: Matrix Algebra}
\author{}
\date{}

\begin{document}
\maketitle
\ifdefined\Shaded\renewenvironment{Shaded}{\begin{tcolorbox}[boxrule=0pt, interior hidden, frame hidden, borderline west={3pt}{0pt}{shadecolor}, breakable, enhanced, sharp corners]}{\end{tcolorbox}}\fi

\hypertarget{q1-compute-a-b}{%
\section{Q1 Compute A + B}\label{q1-compute-a-b}}

\begin{Shaded}
\begin{Highlighting}[]
\NormalTok{A }\OtherTok{\textless{}{-}} \FunctionTok{matrix}\NormalTok{(}
  \FunctionTok{c}\NormalTok{(}\DecValTok{1}\NormalTok{,    }\DecValTok{4}\NormalTok{,   }\DecValTok{2}\NormalTok{,}
    \DecValTok{2}\NormalTok{,    }\DecValTok{0}\NormalTok{,  }\SpecialCharTok{{-}}\DecValTok{5}\NormalTok{,}
    \SpecialCharTok{{-}}\DecValTok{1}\NormalTok{,   }\DecValTok{2}\NormalTok{,   }\DecValTok{1}\NormalTok{,}
    \DecValTok{0}\NormalTok{,    }\DecValTok{1}\NormalTok{,    }\DecValTok{2}
\NormalTok{  ), }\AttributeTok{nrow =} \DecValTok{3}\NormalTok{, }\AttributeTok{ncol=}\DecValTok{4}\NormalTok{)}

\NormalTok{B }\OtherTok{\textless{}{-}} \FunctionTok{matrix}\NormalTok{(}
  \FunctionTok{c}\NormalTok{(}\DecValTok{3}\NormalTok{, }\DecValTok{1}\NormalTok{, }\DecValTok{2}\NormalTok{,}
    \SpecialCharTok{{-}}\DecValTok{4}\NormalTok{, }\DecValTok{5}\NormalTok{, }\SpecialCharTok{{-}}\DecValTok{1}\NormalTok{,}
    \DecValTok{1}\NormalTok{, }\DecValTok{0}\NormalTok{, }\DecValTok{3}\NormalTok{,}
    \DecValTok{2}\NormalTok{, }\DecValTok{3}\NormalTok{, }\SpecialCharTok{{-}}\DecValTok{1}\NormalTok{), }\AttributeTok{nrow=}\DecValTok{3}\NormalTok{, }\AttributeTok{ncol=}\DecValTok{4}
\NormalTok{)}

\NormalTok{A }\SpecialCharTok{+}\NormalTok{ B}
\end{Highlighting}
\end{Shaded}

\begin{verbatim}
     [,1] [,2] [,3] [,4]
[1,]    4   -2    0    2
[2,]    5    5    2    4
[3,]    4   -6    4    1
\end{verbatim}

\[
A + B = \begin{bmatrix}
1 & 2 & -1 & 0 \\
4 & 0 & 2 & 1 \\
2 & -5 & 1 & 2 \\
\end{bmatrix} + \begin{bmatrix}
3 & -4 & 1 & 2 \\
1 & 5 & 0 & 3 \\
2 & -1 & 3 & -1 \\
\end{bmatrix} = \begin{bmatrix}
4 & -2 & 0 & 2 \\
5 & 5 & 2 & 4 \\
4 & -6 & 4 & 1 \\
\end{bmatrix}
\]

\hypertarget{q2-find-h-such-that-a-b---h-0}{%
\section{\texorpdfstring{Q2 Find \(H\) such that
\(A + B - H = 0\)}{Q2 Find H such that A + B - H = 0}}\label{q2-find-h-such-that-a-b---h-0}}

\begin{Shaded}
\begin{Highlighting}[]
\NormalTok{A }\OtherTok{\textless{}{-}} \FunctionTok{matrix}\NormalTok{(}
  \FunctionTok{c}\NormalTok{(}\DecValTok{1}\NormalTok{, }\DecValTok{3}\NormalTok{, }\DecValTok{5}\NormalTok{,}
    \DecValTok{2}\NormalTok{, }\DecValTok{4}\NormalTok{, }\DecValTok{6}
\NormalTok{  ), }\DecValTok{3}\NormalTok{, }\DecValTok{2}\NormalTok{, }\AttributeTok{byrow=}\ConstantTok{FALSE}
\NormalTok{)}

\NormalTok{B }\OtherTok{\textless{}{-}} \FunctionTok{matrix}\NormalTok{(}\FunctionTok{c}\NormalTok{(}
  \SpecialCharTok{{-}}\DecValTok{3}\NormalTok{, }\DecValTok{1}\NormalTok{, }\DecValTok{4}\NormalTok{,}
  \SpecialCharTok{{-}}\DecValTok{2}\NormalTok{, }\SpecialCharTok{{-}}\DecValTok{5}\NormalTok{, }\DecValTok{3}
\NormalTok{),}\DecValTok{3}\NormalTok{, }\DecValTok{2}\NormalTok{, }\AttributeTok{byrow=}\ConstantTok{FALSE}\NormalTok{)}

\NormalTok{H }\OtherTok{\textless{}{-}}\NormalTok{ A }\SpecialCharTok{+}\NormalTok{ B}
\NormalTok{H}
\end{Highlighting}
\end{Shaded}

\begin{verbatim}
     [,1] [,2]
[1,]   -2    0
[2,]    4   -1
[3,]    9    9
\end{verbatim}

\$\$

A + B - H = 0\textbackslash{}

\begin{bmatrix}
1 & 2 \\
3 & 4 \\
5 & 6 \\
\end{bmatrix}

\begin{itemize}
\item
  \begin{bmatrix}
  -3 & -2 \\
  1 & -5  \\
  4 & 3 \\
  \end{bmatrix}

  \begin{itemize}
  \item
    \begin{bmatrix}
    p & q \\
    r & s  \\
    t & u \\
    \end{bmatrix}

    = 0\textbackslash{}

    \begin{bmatrix}
    -2 & 0 \\
    4 & -1  \\
    9 & 9 \\
    \end{bmatrix}

    \begin{itemize}
    \item
      \begin{bmatrix}
      p & q \\
      r & s  \\
      t & u \\
      \end{bmatrix}

      = 0\textbackslash{}

      \begin{bmatrix}
      -2 & 0 \\
      4 & -1  \\
      9 & 9 \\
      \end{bmatrix}

      \begin{itemize}
      \item
        \begin{bmatrix}
        p & q \\
        r & s  \\
        t & u \\
        \end{bmatrix}

        \begin{itemize}
        \item
          \begin{bmatrix}
          p & q \\
          r & s  \\
          t & u \\
          \end{bmatrix}

          = 0 +

          \begin{bmatrix}
          p & q \\
          r & s  \\
          t & u \\
          \end{bmatrix}

          \textbackslash{}

          \begin{bmatrix}
          p & q \\
          r & s  \\
          t & u \\
          \end{bmatrix}

          =

          \begin{bmatrix}
          -2 & 0 \\
          4 & -1  \\
          9 & 9 \\
          \end{bmatrix}
        \end{itemize}
      \end{itemize}
    \end{itemize}
  \end{itemize}
\end{itemize}

\$\$

\hypertarget{q3.-compute-a-b}{%
\section{Q3. Compute A * B}\label{q3.-compute-a-b}}

\begin{Shaded}
\begin{Highlighting}[]
\NormalTok{A }\OtherTok{\textless{}{-}} \FunctionTok{c}\NormalTok{(}\DecValTok{4}\NormalTok{, }\DecValTok{5}\NormalTok{, }\DecValTok{6}\NormalTok{)}
\NormalTok{B }\OtherTok{\textless{}{-}} \FunctionTok{matrix}\NormalTok{(}\FunctionTok{c}\NormalTok{(}\DecValTok{2}\NormalTok{, }\DecValTok{3}\NormalTok{, }\SpecialCharTok{{-}}\DecValTok{1}\NormalTok{), }\DecValTok{3}\NormalTok{, }\DecValTok{1}\NormalTok{)}

\NormalTok{A }\SpecialCharTok{\%*\%}\NormalTok{ B}
\end{Highlighting}
\end{Shaded}

\begin{verbatim}
     [,1]
[1,]   17
\end{verbatim}

\[
\begin{bmatrix}
4 & 5 & 6
\end{bmatrix}
\begin{bmatrix}
2\\3\\-1
\end{bmatrix} =
17
\] \# Q4. Compute A * B

\begin{Shaded}
\begin{Highlighting}[]
\NormalTok{A }\OtherTok{\textless{}{-}} \FunctionTok{matrix}\NormalTok{(}\FunctionTok{c}\NormalTok{(}
  \DecValTok{2}\NormalTok{, }\DecValTok{1}\NormalTok{,}
  \DecValTok{3}\NormalTok{, }\DecValTok{5}\NormalTok{,}
  \DecValTok{4}\NormalTok{, }\DecValTok{6}
\NormalTok{), }\DecValTok{2}\NormalTok{, }\DecValTok{3}\NormalTok{)}
\NormalTok{B }\OtherTok{\textless{}{-}} \FunctionTok{matrix}\NormalTok{(}\FunctionTok{c}\NormalTok{(}\DecValTok{1}\NormalTok{, }\DecValTok{2}\NormalTok{, }\DecValTok{3}\NormalTok{), }\DecValTok{3}\NormalTok{, }\DecValTok{1}\NormalTok{)}

\NormalTok{A }\SpecialCharTok{\%*\%}\NormalTok{ B}
\end{Highlighting}
\end{Shaded}

\begin{verbatim}
     [,1]
[1,]   20
[2,]   29
\end{verbatim}

\hypertarget{q5.-compute-a-b}{%
\section{\texorpdfstring{Q5. Compute
\(A * B\)}{Q5. Compute A * B}}\label{q5.-compute-a-b}}

\begin{Shaded}
\begin{Highlighting}[]
\NormalTok{A }\OtherTok{\textless{}{-}} \FunctionTok{matrix}\NormalTok{(}\FunctionTok{c}\NormalTok{(}
  \DecValTok{1}\NormalTok{, }\DecValTok{1}\NormalTok{, }\DecValTok{1}\NormalTok{,}
  \DecValTok{2}\NormalTok{, }\DecValTok{3}\NormalTok{, }\DecValTok{2}\NormalTok{,}
  \DecValTok{3}\NormalTok{, }\DecValTok{3}\NormalTok{, }\DecValTok{4}
\NormalTok{), }\DecValTok{3}\NormalTok{, }\DecValTok{3}\NormalTok{)}

\NormalTok{B }\OtherTok{\textless{}{-}} \FunctionTok{matrix}\NormalTok{(}\FunctionTok{c}\NormalTok{(}
  \DecValTok{6}\NormalTok{, }\SpecialCharTok{{-}}\DecValTok{1}\NormalTok{, }\DecValTok{1}\NormalTok{,}
  \SpecialCharTok{{-}}\DecValTok{2}\NormalTok{, }\DecValTok{1}\NormalTok{, }\DecValTok{0}\NormalTok{,}
  \SpecialCharTok{{-}}\DecValTok{3}\NormalTok{, }\DecValTok{0}\NormalTok{, }\DecValTok{1}
\NormalTok{), }\DecValTok{3}\NormalTok{, }\DecValTok{3}\NormalTok{)}

\NormalTok{A }\SpecialCharTok{\%*\%}\NormalTok{ B}
\end{Highlighting}
\end{Shaded}

\begin{verbatim}
     [,1] [,2] [,3]
[1,]    7    0    0
[2,]    6    1    0
[3,]    8    0    1
\end{verbatim}

\hypertarget{q6.-compute-a3}{%
\section{\texorpdfstring{Q6. Compute
\(A^3\)}{Q6. Compute A\^{}3}}\label{q6.-compute-a3}}

\begin{Shaded}
\begin{Highlighting}[]
\NormalTok{A }\OtherTok{\textless{}{-}} \FunctionTok{matrix}\NormalTok{(}\FunctionTok{c}\NormalTok{(}
  \DecValTok{2}\NormalTok{, }\SpecialCharTok{{-}}\DecValTok{1}\NormalTok{, }\DecValTok{1}\NormalTok{,}
  \SpecialCharTok{{-}}\DecValTok{2}\NormalTok{, }\DecValTok{3}\NormalTok{, }\DecValTok{2}\NormalTok{,}
  \SpecialCharTok{{-}}\DecValTok{4}\NormalTok{, }\DecValTok{4}\NormalTok{, }\SpecialCharTok{{-}}\DecValTok{3}
\NormalTok{), }\DecValTok{3}\NormalTok{, }\DecValTok{3}\NormalTok{)}

\NormalTok{A }\SpecialCharTok{\%*\%}\NormalTok{ A }\SpecialCharTok{\%*\%}\NormalTok{ A}
\end{Highlighting}
\end{Shaded}

\begin{verbatim}
     [,1] [,2] [,3]
[1,]   18  -66  -68
[2,]  -17   67   68
[3,]    9   26  -35
\end{verbatim}

\hypertarget{q7.-compute-a3}{%
\section{\texorpdfstring{Q7. Compute
\(A^3\)}{Q7. Compute A\^{}3}}\label{q7.-compute-a3}}

\begin{Shaded}
\begin{Highlighting}[]
\NormalTok{A }\OtherTok{\textless{}{-}} \FunctionTok{matrix}\NormalTok{(}\FunctionTok{c}\NormalTok{(}
  \DecValTok{1}\NormalTok{, }\DecValTok{5}\NormalTok{, }\SpecialCharTok{{-}}\DecValTok{2}\NormalTok{,}
  \DecValTok{1}\NormalTok{, }\DecValTok{2}\NormalTok{, }\SpecialCharTok{{-}}\DecValTok{1}\NormalTok{,}
  \DecValTok{3}\NormalTok{, }\DecValTok{6}\NormalTok{, }\SpecialCharTok{{-}}\DecValTok{3}
\NormalTok{), }\DecValTok{3}\NormalTok{, }\DecValTok{3}\NormalTok{)}

\NormalTok{A }\SpecialCharTok{\%*\%}\NormalTok{ A }\SpecialCharTok{\%*\%}\NormalTok{ A}
\end{Highlighting}
\end{Shaded}

\begin{verbatim}
     [,1] [,2] [,3]
[1,]    0    0    0
[2,]    0    0    0
[3,]    0    0    0
\end{verbatim}

\hypertarget{q8.-verify-that-a-b-1-by-computing-ab-and-ba-where}{%
\section{\texorpdfstring{Q8. Verify that \(A = B^{-1}\) by computing
\(AB\) and \(BA\)
where:}{Q8. Verify that A = B\^{}\{-1\} by computing AB and BA where:}}\label{q8.-verify-that-a-b-1-by-computing-ab-and-ba-where}}

\[
\begin{aligned}
A &= B^{-1} \\
AB &= B^{-1}B \\
AB &= I
\end{aligned}
\]

\begin{Shaded}
\begin{Highlighting}[]
\NormalTok{A }\OtherTok{\textless{}{-}} \FunctionTok{matrix}\NormalTok{(}\FunctionTok{c}\NormalTok{(}
  \DecValTok{3}\NormalTok{, }\DecValTok{1}\NormalTok{,}
  \SpecialCharTok{{-}}\DecValTok{1}\NormalTok{, }\SpecialCharTok{{-}}\DecValTok{2}
\NormalTok{), }\DecValTok{2}\NormalTok{, }\DecValTok{2}\NormalTok{)}

\NormalTok{B }\OtherTok{\textless{}{-}} \FunctionTok{matrix}\NormalTok{(}\FunctionTok{c}\NormalTok{(}
\NormalTok{  .}\DecValTok{4}\NormalTok{, .}\DecValTok{2}\NormalTok{,}
  \SpecialCharTok{{-}}\NormalTok{.}\DecValTok{2}\NormalTok{, }\SpecialCharTok{{-}}\NormalTok{.}\DecValTok{6}
\NormalTok{), }\DecValTok{2}\NormalTok{, }\DecValTok{2}\NormalTok{)}

\NormalTok{A }\SpecialCharTok{\%*\%}\NormalTok{ B}
\end{Highlighting}
\end{Shaded}

\begin{verbatim}
     [,1]          [,2]
[1,]    1 -1.110223e-16
[2,]    0  1.000000e+00
\end{verbatim}

\begin{Shaded}
\begin{Highlighting}[]
\NormalTok{B }\SpecialCharTok{\%*\%}\NormalTok{ A}
\end{Highlighting}
\end{Shaded}

\begin{verbatim}
             [,1] [,2]
[1,] 1.000000e+00    0
[2,] 1.110223e-16    1
\end{verbatim}

\hypertarget{compute-a}{%
\subsection{\texorpdfstring{Compute \(A\)}{Compute A}}\label{compute-a}}

\begin{Shaded}
\begin{Highlighting}[]
\NormalTok{A }\OtherTok{\textless{}{-}} \FunctionTok{matrix}\NormalTok{(}\FunctionTok{c}\NormalTok{(}
  \DecValTok{1}\NormalTok{, }\DecValTok{4}\NormalTok{, }\DecValTok{2}\NormalTok{,}
  \DecValTok{2}\NormalTok{, }\DecValTok{0}\NormalTok{ ,}\SpecialCharTok{{-}}\DecValTok{5}\NormalTok{,}
  \SpecialCharTok{{-}}\DecValTok{1}\NormalTok{, }\DecValTok{2}\NormalTok{, }\DecValTok{1}\NormalTok{,}
  \DecValTok{0}\NormalTok{, }\DecValTok{1}\NormalTok{, }\DecValTok{2}
\NormalTok{), }\DecValTok{3}\NormalTok{, }\DecValTok{4}\NormalTok{)}

\FunctionTok{t}\NormalTok{(A)}
\end{Highlighting}
\end{Shaded}

\begin{verbatim}
     [,1] [,2] [,3]
[1,]    1    4    2
[2,]    2    0   -5
[3,]   -1    2    1
[4,]    0    1    2
\end{verbatim}

\hypertarget{compute-tra}{%
\subsection{\texorpdfstring{Compute
\(tr(A)\)}{Compute tr(A)}}\label{compute-tra}}

\[
tr(A)=\sum_{i=1}^n a_{ii}
\]

\begin{Shaded}
\begin{Highlighting}[]
\NormalTok{A }\OtherTok{\textless{}{-}} \FunctionTok{matrix}\NormalTok{(}\FunctionTok{c}\NormalTok{(}
  \DecValTok{1}\NormalTok{, }\DecValTok{3}\NormalTok{,}
  \DecValTok{2}\NormalTok{, }\DecValTok{4}
\NormalTok{), }\DecValTok{2}\NormalTok{, }\DecValTok{2}\NormalTok{) }

\FunctionTok{sum}\NormalTok{(}\FunctionTok{diag}\NormalTok{(A))}
\end{Highlighting}
\end{Shaded}

\begin{verbatim}
[1] 5
\end{verbatim}

\hypertarget{compute-a-1}{%
\subsection{\texorpdfstring{Compute
\(|A|\)}{Compute \textbar A\textbar{}}}\label{compute-a-1}}

\[
\begin{aligned}
|A| &= (a * d) - (b * c)\\
|A| &= (1*4) - (2 * 3)\\
|A| &= 4-6\\
|A| &=-2
\end{aligned}
\]

\begin{Shaded}
\begin{Highlighting}[]
\NormalTok{A }\OtherTok{\textless{}{-}} \FunctionTok{matrix}\NormalTok{(}\FunctionTok{c}\NormalTok{(}
  \DecValTok{1}\NormalTok{, }\DecValTok{3}\NormalTok{,}
  \DecValTok{2}\NormalTok{, }\DecValTok{4}
\NormalTok{), }\DecValTok{2}\NormalTok{, }\DecValTok{2}\NormalTok{)}
\FunctionTok{det}\NormalTok{(A)}
\end{Highlighting}
\end{Shaded}

\begin{verbatim}
[1] -2
\end{verbatim}



\end{document}
